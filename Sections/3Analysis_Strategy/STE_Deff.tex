
{\bf \normalsize (i) Tracking efficiency} is calculated by obtaining the ratio between the yield at the reconstructed level and generated level, for a defined ``type" of particles and it is estimated differentially in p$_T$, $\eta$, and z$_{vtx}$ of the event.\\
{\bf Implementation }: tracking efficiency maps are produced as TH3D histograms (p$_T$, $\eta$, z$_{vtx}$) obtained from MC analysis after all event and particles/track selections (summarized in Table.~\ref{table:effCuts}). These efficiency maps are used in the analysis tasks to extract single track efficiencies in which each associated track is inserted in correlation plots with a weight of {\bf 1/efficiency value}. Example plots of the tracking efficiencies as a function of p$_T$ are shown in Fig.~\ref{fig:trackeff} for different particle species. %for different set of DCA values and Fig.~\ref{fig:trackeffvsspecies} for the different particle species.
The one-dimension p$_\mathrm{T}$ efficiency with different sets of track cuts is almost flat at mid and high p$_\mathrm{T}$ range and has small fluctuation in the low p$_\mathrm{T}$ region and average efficiency per species in mentioned in Table.~\ref{table:AvgeffValue}.

%\begin{figure}[h]
%	\centering
%	\includegraphics[scale=0.35]{figures/pPb_STE_1D_pT_all_DCA.png}
%	\caption{Charged particle $p_T$ efficiency for different DCA values.}
%	\label{fig:trackeff}	
%\end{figure}


\begin{figure}[h]
	\centering
	%\includegraphics[scale=0.35]{figures/pPb_efficiencyVsspeicies.png}
	\caption{$p_T$ efficiency for different species.}
	\label{fig:trackeffvsspecies}	
\end{figure}


% new table on Nov3 for average efficiency
\begin{table}[h]
\small
\centering % used for centering table
\begin{tabular}{ p{3.5cm} | p{2.5cm} | p{2.5cm} |  p{2.5cm}  }
{\normalsize \textbf {Species / $p_t$ range }} & {\normalsize \textbf {0.3-24.0 GeV/c}} &   {\normalsize \textbf {0.5-24.0 GeV/c}} &   {\normalsize \textbf {0.3-0.5 GeV/c}}\\
\hline
&&\\		
all-Charge & 0.832211 & 0.850089 & 0.804515 \\
Pions & 0.865529 & 0.889492 & 0.834333 \\
Kaons & 0.678151 & 0.729512 & 0.528229 \\
Protons & 0.853865 & 0.883295 & 0.714668 \\
Electrons & 0.869452 & 0.856509 & 0.903016 \\
 \hline \hline
\end{tabular}
\caption{\large {Average $p_T$ efficiency per species (LHC16q/pass1$\_$woSDD/AOD)}} % title of Table
\label{table:AvgeffValue}	
\end{table}

\newpage
Details of cuts at event level and particle selection at different steps are listed in Table.~\ref{table:effCuts} . \\
\begin{table}[h]
\small
\centering % used for centering table

\begin{tabular}{ p{4cm} c | p{5cm} |  p{5.5cm} }
 \\
  \multirow{1}{*}{\large \textbf {MC Generated }} \\
\hline
\\
  {\normalsize \textbf {Event Selection}} &&               \multicolumn{1}{c}{{\normalsize \textbf {Particle Selection}}} \\
		            &&               Stages         &              Cuts \\
\hline\hline && &\\		            	
1. -10cm $\textless$ Zvtx $\textless$ 10cm  &&               1.MC Part with Generated Cuts         &    {\textbf {After Event Selection}}\\
																		   &&&Charge\\
																		    &&& PDG Code\\
														  				  &&& Physical Primary \\


										
2. Multiplicity                     &&               2. MC Part with Kine Cuts         &              {\textbf {Kinematics Cuts }}\\
															    &&& -0.8$\textless \eta \textless  0.8$\\
															    &&& 0.3 $\textless$ pT $\textless $ 24 (GeV/$c$)\\


3. Type of MC process    &&               3. MC Part with Acceptance Cuts         &      {\textbf {nhits ref on detectors ( ESD only)}}  \\
															      &&& a) ITS = 4 \\
														 	      &&& b) TPC = 5 \\	
														                &&& c) TOF = 0\\
															       &&& d) MOUN = 0\\
&& &		\\            	


\multirow{1}{*}{\large \textbf {MC Reconstructed }} &&& \\
\hline


\hline && &\\		            	                        	
1. -10cm $\textless$ Zvtx $\textless$ 10cm  &&               4. Reco tracks        &                             {\textbf {After Event Selection}}\\
															    &&& Physical Primary \\
															
															
2. Multiplicity                     &&               5. Reco tracks with Kine Cuts         &               {\textbf  {Kinematics Cuts }}\\
															    &&& -0.8$\textless \eta \textless  0.8$\\
															    &&& 0 $\textless$ pT $\textless $ 24 (GeV/$c$)\\



3. Physics Selection    &&               6. MC true with Quality Cuts         &      			      {\textbf  {Quality Cuts }} \\
																	&&&SetRequireSigmaToVertex(kFALSE) \\
																	&&&SetDCAToVertex2D(kFALSE) \\
																	&&&SetMinNClustersTPC(70)\\
																	&&&SetMinNClustersITS(3)\\
																	&&&SetMaxChi2PerClusterTPC(4)\\
																	&&&SetMaxDCAToVertexZ(1) \\
																	&&&SetMaxDCAToVertexXY(0.25) \\
																	&&&SetRequireTPCReï¿œt(TRUE) \\
																	&&&SetRequireITSReï¿œt(TRUE) \\
																	&&&SetClusterRequirementITS(SPD, ..) \\


 				      &&               7. Reco tracks with Quality Cuts         &             {\textbf  {Same as step 6}} \\

&& &\\		            	            		

 \hline \hline
 \\
\end{tabular}
\caption{\large {Single Track Efficiency cuts detail}} % title of Table
\label{table:effCuts}	
\end{table}


{\bf \large (ii) D Meson efficiency} - Due to limited statistics, the correlation analysis is performed in wide p$_\mathrm{T}$ bins and in each of them the reconstruction and selection efficiency of D mesons is not flat (Fig.~\ref{fig:dpluseff},~\ref{fig:d0eff}). We correct for the p$_\mathrm{T}$ dependence of the trigger efficiency within each p$_\mathrm{T}$-bin.
This correction is applied online, by using a map of D meson efficiency as a function of p$_\mathrm{T}$ and event multiplicity extracted from MC.
While running the analysis, each correlation entry is weighted by {\bf 1/trigger efficiency}. It was observed that multiplicity dependence of the efficiency does not bias the extraction of the signal yield from the invariant mass distributions. Efficiency plots for $D^{+}$ meson and $D^{0}$ meson are shown in Fig.~\ref{fig:dpluseff} and ~\ref{fig:d0eff}.

\begin{figure}[h]
	\centering
	%Marianna
	%\includegraphics[width=.50\linewidth]{figures/Efficiency_Dplus_Corrected_Central.png} \\% by Jitendra
	%\includegraphics[width=.30\linewidth]{figures/Dplus_3_5Multplicty_eff.png}
	%\includegraphics[width=.30\linewidth]{figures/Dplus_5_8Multplicty_eff.png}
	%\includegraphics[width=.30\linewidth]{figures/Dplus_8_16Multplicty_eff.png}
	\caption{Top panel: (p$_T$, multiplicity) dependence of $D^+$ meson efficiency. Bottom panels: $D^+$ meson efficiency in multiplicity for three $D^+$ p$_\mathrm{T}$ranges: 3-5 GeV/$c$ (left), 5-8 GeV/$c$ (center), 8-16 GeV/$c$ (right).}
	\label{fig:dpluseff}	
\end{figure}

\begin{figure}[!htp]
	\centering
%Marianna
	%\includegraphics[width=.48\linewidth]{figures/D0Eff_From_c_wLimAcc_2D_pPb.png}  % by Fabio
	%\includegraphics[width=.48\linewidth]{figures/D0Eff_From_c_wLimAcc_1D_pPb.png} \\
	%\includegraphics[width=.30\linewidth]{figures/D0Eff_ProjMult_3to4GeV.png}
	%\includegraphics[width=.30\linewidth]{figures/D0Eff_ProjMult_5to6GeV.png}
	%\includegraphics[width=.30\linewidth]{figures/D0Eff_ProjMult_8to12GeV.png}
	\caption{Top panel: (p$_\mathrm{T}$, multiplicity) dependence (left) and p$_T$ dependence (right) of prompt $D^0$ meson efficiency. Bottom panels: multiplicity dependence of $D^0$ meson efficiency for three $D^0$ p$_\mathrm{T}$ ranges: 3-4 GeV/$c$ (left), 5-6 GeV/$c$ (center), 8-12 GeV/$c$ (right). For tracklet multiplicity$>$ 120, due to the limited statistics, the efficiency value is fixed to the one obtained for 90$<$tracklet multiplicity$<$120.}
	\label{fig:d0eff}	
\end{figure}
\newpage 